%----------------------------------------------------------------
%
%  File    :  survey.tex
%
%  Author  :  Keith Andrews, ISDS, TU Graz, Austria
%
%  Created :  24 Mar 2010
%
%  Changed :  22 Nov 2016
%
%----------------------------------------------------------------


\documentclass[11pt,onecolumn,twoside]{report}

\usepackage[          % set page and margin sizes
  a4paper,
  twoside,
  top=5mm,
  bottom=10mm,
  inner=15mm,
  outer=15mm,
  bindingoffset=10mm,
  head=10mm,
  foot=10mm,
  headsep=15mm,
  footskip=15mm,
  includeheadfoot,
]{geometry}
% A4 is 210 x 297 mm



\usepackage{txfonts}            % new times fonts
\usepackage{relsize}            % relative font sizes \smaller \larger

\usepackage[T1]{fontenc}        % 8-bit output chars (must be before inputenx)
\usepackage[utf8]{inputenx}     % input char encoding

\usepackage{float}              % H for float placement
\usepackage{setspace}           % line spacing

\usepackage{textcomp}           % symbols such as \texttimes and \texteuro
\usepackage{latexsym}

\usepackage{siunitx}            % prettier number formatting
\sisetup{%
  group-separator={,},
}
\usepackage[super]{nth}         % 1st, 2nd, 3rd, etc.

\usepackage{xspace}
\usepackage{etoolbox}           % for \newrobustcmd
\usepackage{makecmds}           % for \makecommand
\usepackage{placeins}		   % for FloatBarrier command


\usepackage[english,austrian,british]{babel}

\usepackage{tabu}

\usepackage[bf,sf]{titlesec}



\setlength{\textfloatsep}{10mm plus 2mm minus 1mm}
\setlength{\floatsep}{10mm plus 2mm minus 1mm}
\setlength{\intextsep}{10mm plus 2mm minus 1mm}

\setlength{\dbltextfloatsep}{10mm plus 2mm minus 1mm}
\setlength{\dblfloatsep}{10mm plus 2mm minus 1mm}

\setlength{\abovecaptionskip}{4mm plus 2mm minus 1mm}
\setlength{\belowcaptionskip}{0mm}



% sensible settings for floats
% See http://www-rohan.sdsu.edu/~aty/bibliog/latex/floats.html
% See https://robjhyndman.com/hyndsight/latex-floats/

\setcounter{topnumber}{2}               % max num floats at top of page
\setcounter{dbltopnumber}{2}            % max num floats on 2col page
\setcounter{bottomnumber}{2}            % max num floats at bottom of page
\setcounter{totalnumber}{4}             % max num floats on a page

\renewcommand{\topfraction}{0.8}        % max fraction of floats at top
\renewcommand{\dbltopfraction}{0.9}     % max fraction of floats at top 2col
\renewcommand{\bottomfraction}{0.8}     % max fraction of floats at bottom
\renewcommand{\textfraction}{0.2}       % min fraction of text

% only for entirely float pages:
\renewcommand{\floatpagefraction}{0.7}      % min page fraction having floats
\renewcommand{\dblfloatpagefraction}{0.7}   % min 2col page fraction having floats






% use caption and subfig (caption2 and subfigure are now obsolete)

\usepackage[
  position=bottom,
  margin=1cm,
  font=small,
  labelfont={bf,sf},
  format=hang,
  indention=0mm,
]{caption,subfig}

\captionsetup[subfigure]{
  margin=0pt,
  parskip=0pt,
  hangindent=0pt,
  indention=0pt,
  singlelinecheck=true,
}




% fancyhdr to make nice headers and footers
% and deal with long chapter names

\usepackage{fancyhdr}         % headers and footers
\pagestyle{fancy}             % must call to set defaults before redefining

\renewcommand{\chaptermark}[1]{%
  \markboth{\thechapter.\ #1}{}
}
\renewcommand{\sectionmark}[1]{%
  \markright{\thesection.\ #1}
}
\renewcommand{\headrulewidth}{0mm}
\renewcommand{\footrulewidth}{0mm}
\newcommand{\headlook}{\sffamily}
\fancyhf{}
\fancyhead[LE,RO]{\thepage}
\fancyhead[LO]{%
\parbox[t]{0.8\textwidth}{\headlook\nouppercase{\rightmark}}
}
\fancyhead[RE]{%
\parbox[t]{0.8\textwidth}{\raggedleft\headlook\nouppercase{\leftmark}}
}


%\fancypagestyle{plain}{%   redefine plain style, but doesn't work
%  \fancyhf{}    % clear all header and footer fields
%  \fancyfoot[C]{\headlook \thepage} % except the center
%  \renewcommand{\headrulewidth}{0pt}
%  \renewcommand{\footrulewidth}{0pt}
%}



\usepackage{xcolor}
\definecolor{darkgreen}{rgb}{0.0,0.2,0.0}
\definecolor{darkblue}{rgb}{0.0,0.0,0.2}
\definecolor{darkred}{rgb}{0.2,0.0,0.0}
\definecolor{verylightgrey}{gray}{0.95}
\definecolor{lightgrey}{gray}{0.9}
\definecolor{black}{gray}{0.0}


\usepackage{tabularx}


\usepackage{listings}                 % for listings of source code
\usepackage{calc}                     % for calculation below

\makeatletter
\newlength{\numwidth}%
\setlength{\numwidth}{\widthof{\normalfont{\lst@numberstyle{99}}}}% Up to 2-digit (99) line numbers
\def\lst@PlaceNumber{%
  \makebox[\numwidth+1em][l]{%
    \makebox[\numwidth][r]{\normalfont\lst@numberstyle{\thelstnumber}}%
  }%
}
\makeatother

\lstset{                              % set parameters for listings
  floatplacement=tp,                  % default float placement
  numberbychapter,
  inputencoding=utf8,
  language=,                          % empty = plain text
  basicstyle=\small\ttfamily,
  tabsize=2,
  xleftmargin=1cm,
  xrightmargin=1cm,
  frame=none,
  framexleftmargin=0mm,
  rulesepcolor=\color{verylightgrey},
  numbers=none,
  numberstyle=\scriptsize,
  numbersep=2ex,
  breaklines,
  showtabs=false,
  showspaces=false,
  showstringspaces=false,
  keywordstyle=\bfseries,
  identifierstyle=,
  stringstyle=,
  captionpos=b,
  abovecaptionskip=\abovecaptionskip,
  belowcaptionskip=\belowcaptionskip,
  aboveskip=\floatsep,
  belowskip=\floatsep,
  extendedchars=true,
  literate=%
    {©}{{\textcopyright}}1
    {€}{{\texteuro}}1
    {Ö}{{\"O}}1
    {Ä}{{\"A}}1
    {Ü}{{\"U}}1
    {ß}{{\ss}}1
    {ö}{{\"o}}1
    {ä}{{\"a}}1
    {ü}{{\"u}}1,       % map some utf8 chars to replacements
}


\lstdefinelanguage{biblatex}   % based on biblatex v 2.7a from 2013-07-14
{
  keywords={%
    @article,@book,@mvbook,@inbook,@bookinbook,@suppbook,%
    @booklet,@collection,@mvcollection,@incollection,@suppcollection,%
    @manual,@misc,@online,@patent,@periodical,@suppperiodical,%
    @proceedings,@mvproceedings,@inproceedings,@reference,@mvreference,%
    @inreference,@report,@set,@thesis,@unpublished,@xdata,%
    @conference,@electronic,@mastersthesis,@phdthesis,@techreport,@www,%
    @artwork,@audio,@bibnote,@commentary,@image,@jurisdiction,@legislation,%
    @legal,@letter,@movie,@music,@performance,@review,@software,%
    @standard,@video%
  },
  comment=[l][\itshape]{@comment},
  sensitive=false,
}


\usepackage[short]{datetime}   % load datetime *after* babel, requires fmtcount
% \newdateformat{britdate}{%
% \ordinaldate{\THEDAY} \,\monthname[\THEMONTH] \THEYEAR
% }
\newdateformat{keithdate}{%
\twodigit{\THEDAY}~\shortmonthname[\THEMONTH]~\THEYEAR
}


\usepackage[hyphens,obeyspaces]{url}
\def\UrlFont{\smaller\ttfamily}



\usepackage[
  autostyle,
  english=british,
  threshold=0,
  thresholdtype=lines,
]{csquotes}
\renewcommand{\mkcitation}[1]{\space#1}

\newenvironment*{smallquote}          % smaller text within a block quote
  {\quote\smaller}
  {\endquote}
\SetBlockEnvironment{smallquote}

% put quotation marks around block quotes
% \renewenvironment{quoteblock}{\openautoquote}{\closeautoquote}

% I prefer double quotes as outer
\DeclareQuoteStyle[keithbritish]{british}%  [variant]{style}
  {\textquotedblleft}%                      opening outer mark
  {\textquotedblright}%                     closing outer mark
  [0.05em]%
  {\textquoteleft}%                         opening inner mark
  {\textquoteright}%                        closing inner mark

\setquotestyle[keithbritish]{british}



\usepackage[
  backend=biber,
  bibstyle=authoryear-ka,
  citestyle=authoryear-ka,
  sorting=nyt,
  useprefix,                   % van and von are part of second name
  mergedate=false,             % only for authoryear style
  dashed=false,                % only for authoryear style
  abbreviate=false,
  maxcitenames=2,              % if more than two authors, then use et al
  mincitenames=1,              % if exceeds 2 authors, then use 2
  maxbibnames=99,              % print all authors in biblio
  uniquename=init,
  hyperref=true,
  backref=true,
  backrefstyle=two,
  natbib=true,
  sortlocale=en,
]{biblatex}



% set for csquotes, but \autocite only available
% after biblatex is loaded
\SetCiteCommand{\autocite}    % or maybe \parencite

% more space between entries in bib
\setlength\bibitemsep{1.5\itemsep}


% remove URL: from in front of URLs
\DeclareFieldFormat{url}{\url{#1}}
\DeclareFieldFormat{doi}{\doi{#1}}
\DeclareFieldFormat{isbn}{\isbn{#1}}
\DeclareFieldFormat{issn}{\issn{#1}}

% suppress urldate field
\DeclareSourcemap{
  \maps[datatype=bibtex]{
    \map{
      \step[fieldset=urldate, null]
    }
  }
}

% for article titles
\DeclareFieldFormat{title:article}{\emph{#1}\midsentence}

\DefineBibliographyStrings{british}{%
  january          = {Jan},
  february         = {Feb},
  march            = {Mar},
  april            = {Apr},
  may              = {May},
  june             = {Jun},
  july             = {Jul},
  august           = {Aug},
  september        = {Sep},
  october          = {Oct},
  november         = {Nov},
  december         = {Dec},
}



% \bibliography{kandrews,latex,writing,inm-plag}

\addbibresource{writing.bib}
\addbibresource{latex.bib}
\addbibresource{kandrews.bib}
\addbibresource{ivis.bib}




\usepackage{ifpdf}

\ifpdf
  % pdflatex
  \usepackage[pdftex]{graphicx}
  \DeclareGraphicsExtensions{.pdf,.jpg,.png}
  \pdfcompresslevel=9
  \pdfpageheight=297mm
  \pdfpagewidth=210mm
  \usepackage[         % hyperref should be last package loaded
    unicode,
    pdftex,
    pdftitle={Writing a Survey Paper},
    pdfsubject={},
    pdfauthor={Eva Rott, Michael Glatzhofer, Dominik Mocher, Julian Wolf},
    pdfkeywords={Master's Thesis, skeleton, guidelines, template},
    bookmarks,
    bookmarksnumbered,
    linktocpage,
    colorlinks,
    linkcolor=darkred,
    anchorcolor=red,
    citecolor=darkgreen,
    urlcolor=darkblue,
    pdfview={FitH},
    pdfstartview={Fit},
    pdfpagemode=UseOutlines,       % open bookmarks in Acrobat
    plainpages=false,              % avoids duplicate page number problem
    pdfpagelabels,                 % avoids duplicate page number problem
    breaklinks=true,               % allow links exceeding a single line
  ]{hyperref}

\else
  % latex
  % should never have to run latex, since l2h now understands pdflatex .aux
  \usepackage[dvips]{graphicx}
  \usepackage[dvips]{hyperref}
  \DeclareGraphicsExtensions{.eps}
\fi





% \liintro list item intro is a style used when list items have an
% introduction phrase (say in italics) followed by a colon.
\newcommand{\liintro}[1]{\emph{#1}}


\newcommand{\imgcredit}[1]
{\smaller{}[#1]}



\newcommand{\copyrightACM}
{%
Copyright \copyright\ by the Association for Computing Machinery, Inc.%
}




\newcommand{\daymonthyear}[3]
{%
\twodigit{#1}\hspace{0.7ex}\nolinebreak[2]\shortmonthname[#2]\hspace{0.7ex}\nolinebreak[2]#3%
}


\newcommand{\monthyear}[2]
{%
\shortmonthname[#1]\hspace{0.7ex}\nolinebreak[2]#2%
}


\newcommand{\yearmonthday}[3]
{%
\twodigit{#3}\hspace{0.7ex}\nolinebreak[2]\shortmonthname[#2]\hspace{0.7ex}\nolinebreak[2]#1%
}


\newcommand{\yearmonth}[2]
{%
\shortmonthname[#2]\hspace{0.7ex}\nolinebreak[2]#1%
}



% link to Amazon or
% http://worldcatlibraries.org/wcpa/isbn/[ISBN number]

\newrobustcmd{\isbn}[1]
{%
{%
\ifpdf
{\smaller ISBN}
\href{http://www.amazon.com/exec/obidos/ASIN/#1/keithandrewshcic}{#1}%
\else
{\smaller ISBN}
#1%
\fi
}%
}



% ISSN
% http://www.bl.uk/services/bibliographic/issn.html
% 8 digits, should be printed xxxx-xxxx
% e.g. 0020-0190 is Information Processing Letters, Elsevier
%
% Lookup services:
% http://kmittlib.lib.kmutt.ac.th:81/search/i?SEARCH=0020-0190
% http://worldcatlibraries.org/wcpa/issn/0020-0190

\newrobustcmd{\issn}[1]
{%
{%
\ifpdf
{\smaller ISSN}
\href{http://worldcatlibraries.org/wcpa/issn/#1}{#1}%
\else
{\smaller ISSN}
#1%
\fi
}%
}



% DOIs  http://www.doi.org/  e.g.
% doi:10.1038/nature723
% http://dx.doi.org/10.1038/nature723

\newrobustcmd{\doi}[1]
{%
{%
\def\UrlFont{\rmfamily}
\ifpdf                                   % pdflatex
\href{http://dx.doi.org/#1}{doi:\protect\nolinkurl{#1}}%
\else                                    % latex
doi:\protect\nolinkurl{#1}%
\fi
}%
}





\newrobustcmd{\website}[1]
{%
\ifpdf                                  % pdflatex
\href{http://#1/}{\nolinkurl{#1}}%
\else                                   % latex
\nolinkurl{#1}%
\fi
}




\newcommand{\news}[1]
{%
\ifpdf
\href{news:#1}{\nolinkurl{#1}}
\else
\nolinkurl{#1}%
\fi
}








% based on url package
% define styles for class, file, and variable names
% which break nicely at line breaks

% make the macros robust so they work inside captions, etc

\newcommand{\ttname}{\begingroup \smaller\urlstyle{tt}\Url}
\newcommand{\rmname}{\begingroup \smaller\urlstyle{rm}\Url}
\newcommand{\sfname}{\begingroup \smaller\urlstyle{sf}\Url}


% cname is for class names
\newrobustcmd{\cname}[1]{\sfname{#1}}

% fname is for file names and directory names
\newrobustcmd{\fname}[1]{\ttname{#1}}

% vname is for variable names, domain names, email addresses
\newrobustcmd{\vname}[1]{\ttname{#1}}



% Euro symbol
\newcommand{\euro}{\texteuro\,}

% times symbol
\newcommand{\timessym}{\texttimes\,}

% approx symbol
\newcommand{\approxsym}{\ensuremath\approx\,}

% plusminus symbol
\newcommand{\plusminussym}{\textpm\,}

% not equal symbol
\newcommand{\neqsym}{\ensuremath{\neq\,}}

% rightarrow symbol
\newcommand{\rightarrowsym}{\ensuremath\rightarrow\,\,}




\newcommand{\TODO}[1]
{
{\textcolor{red}{[TODO: #1]}}
}



\newcommand{\fullh}{18cm}         % height of figures for 1 per page
\newcommand{\halfh}{9.5cm}        % height of figures for 2 per page
\newcommand{\thirdh}{6cm}         % height of figures for 3 per page


\tolerance=400 
  % makes some lines with lots of white space, but      
  % tends to prevent words from sticking out in the margin





\begin{document}

\keithdate

\normalsize
\pagestyle{empty}         % for preliminary pages (no numbers shown)
\pagenumbering{Roman}     % for pdf labels




\begin{titlepage}

\begin{center}
{\Large \sffamily \bfseries Writing a Survey Paper}

\vspace{1cm}

{\large\sffamily Eva Rott\\ Michael Glatzhofer\\ Dominik Mocher\\ Julian Wolf}

% {\large\sffamily Group 4}
% \vspace{5mm}
% {\large\sffamily Keith Andrews, Tom Strong, Bill Weak, and Seb Green}

\vspace{1cm}

Institute of Interactive Systems and Data Science (ISDS), \\
Graz University of Technology \\
A-8010 Graz, Austria \\[1cm]


% {\large
% 706.057 Information Visualisation SS 2016 \\
% Graz University of Technology \\[1cm]
% }

\vspace{1cm}

{\large 18 May 2017}

\end{center}



\vspace{2cm}

\begin{quote}
\begin{center}
{\large\sffamily\bfseries Abstract}
\end{center}

By representing graphs in an adjancy matrix it is possible to observe special patterns and reveal dependencies which might not be seen in the graph per se. By combining the benefits of both the matrix representation and the graph itself, a very powerful approach of graph analysis may be achieved. 

In this survey we present some potent techniques applicable to adjacency matrices to analyze graphs. Furthermore, we present some tools utilizing these techniques.

\end{quote}

\vfill

\begin{center}
{\small\sffamily \copyright ~ Copyright 2016 by the author(s),
except as otherwise noted.}

\vspace{2mm}
{\footnotesize\sffamily This work is placed under a
Creative Commons Attribution 4.0 International
(\href{https://creativecommons.org/licenses/by/4.0/}{CC BY 4.0}) licence.}
\end{center}

\end{titlepage}




\cleardoublepage
\pagestyle{plain}
\pagenumbering{roman}



{
\setlength{\parskip}{3pt plus 3pt minus 3pt}     % compact tables of contents
\tableofcontents
\addcontentsline{toc}{chapter}{Contents}

\cleardoublepage
\listoffigures
\addcontentsline{toc}{chapter}{List of Figures}

\cleardoublepage
\listoftables
\addcontentsline{toc}{chapter}{List of Tables}

\cleardoublepage
\renewcommand{\lstlistlistingname}{List of Listings}
\lstlistoflistings
\addcontentsline{toc}{chapter}{List of Listings}
}


\cleardoublepage
\pagestyle{headings}        % for main pages
\pagenumbering{arabic}


\cleardoublepage
%----------------------------------------------------------------
%
%  File    :  survey-basics.tex
%
%  Author  :  Julian Philipp Wolf, TU Graz, Austria
% 
%  Created :  14 May 2017
% 
%  Changed :  16 May 2017
% 
%----------------------------------------------------------------


\chapter{Basics}\label{chap:Basics}


This chapter explains the different types of graphs and their corresponding matrices, which techniques are used on matrices and how to interpret the resulting patterns.

\section{Definitions}
In mathematics a graph is an ordered pair $G = (V, E)$ containing a set of nodes $V$ and a set of edges $E$. However, some literature refer to nodes as ``vertices" (thus the $V$) or ``points".
Edges may be called ``arc" or lines. 
On the other hand, in the case of an directed graph, edges may also be called arrows. Moreover: \begin{itemize}
	\item $V$ is not allowed to be empty
	\item $E$ is allowed to be empty
	\item The \textbf{order} of a graph is the number of vertices $|V|$
	\item The \textbf{size} of a graph is the number of edges $|E|$
\end{itemize}

In this paper, every node in the graph has its distinctive unique id, which never changes. This holds for the reordering of the matrices too - when reordering rows and columns, the corresponding index stays with the column, otherwise the graph would be changed with this operation.


\section{Types of Graphs}

Basically, there are two types of edges (directed and undirected) and two types of cost calculations (weighted and unweighted), which leads to 4 different graphs. Figure~\ref{fig:graph_types} shows these 4 different types of graphs. 


\begin{table}[H]
  \centering
\begin{tabu}{cc}
	\includegraphics[width=0.49\textwidth]{images/directed_weighted_graph}
	 &
	\includegraphics[width=0.49\textwidth]{images/undirected_weighted_graph} \\
	\includegraphics[width=0.49\textwidth]{images/directed_graph}  &
	\includegraphics[width=0.49\textwidth]{images/undirected_graph} 
\end{tabu}
\captionof{figure}{4 different types of graphs (top: weighted directed and undirected, bottom: unweighted directed and undirected)\label{fig:graph_types}}
\end{table}


\subsection{directed/undirected}

Undirected edges may be traversed in any direction, whereas directed edges may just be traversed in one direction. For matrix representation of graphs, neither the mathematical \textbf{quiver}, a directed graph which has multiple arrows pointing from node $x$ to $y$, nor the \textbf{multigraph}, which is a graph which contains multiple undirected edges connecting just two nodes, is used.


\subsection{weighted/unweighted}

For graphs without costs defined for their edges, so called \textbf{unweighted graphs}, may be processed differently: 
\begin{itemize}
\item Either the algorithm searches for the shortest path, thus defining a uniform cost on all edges
\item Or there is no cost calculation at all, even the amount of edges is ignored
\end{itemize}

When adding weights to the edges, the graph is called a \textbf{weighted graph}. These weights typically represent different things, for example:
\begin{itemize}
\item time
\item length
\item energy consumption
\item elevations
\end{itemize}
to name just a few.

With weights defined on the edges, the approaches of algorithms are different, as the shortest path, in despite of number of traversed edges, is not necessarily the cheapest one. 



\section{Use cases}
Some use cases of the different graph types are (to name just a few examples):
\begin{itemize}
\item Navigation system (weighted directed)
	\begin{itemize}
		\item Nodes: Cities/POIs
		\item Edges: Routes directed (one way streets)
		\begin{itemize}
			\item weights
			\begin{itemize}
				\item length of street (find shortest way)
				\item time to traverse the street (find fastest way)
			\end{itemize}		
		\end{itemize}
	\end{itemize}		
\item Subway map (undirected unweighted)
\begin{itemize}
	\item Nodes: stations
	\item edges: connection between stations
\end{itemize}
\item Relations of tweets (directed unweighted)
\begin{itemize}
	\item nodes: single tweet entry
	\item edges: references to other tweets
\end{itemize}
\end{itemize}


\section{Matrix representation of graphs}

When representing graphs in a matrix, an adjacency matrix is used. Adjacency matrices are structured with every row and every column represents one node. This leads to a N x N square matrix, where N is the number of nodes. 

These matrices show some patterns according to their corresponding graph but most times these patterns are not immediately visible. There are some techniques to reveal these patterns, all of them involving the reordering of the matrix.


\subsection{Reordering}

The main goal of reordering the matrix is to cluster the edges and thus reveal certain patterns. An example of this behaviour can be seen in figure~\ref{fig:reorder}. 

\begin{figure}[h]
\includegraphics[width=\textwidth]{images/reorder}
\caption{Reordering a matrix\label{fig:reorder}}
\end{figure}


When reordering the matrix, the indices of the single rows and columns stay with the rows, otherwise the graph would change by this workstep. In this example, at first the rows 1 and 4 get swapped and as a second step columns 2 and 4. In this way the full connection pattern of the two subgraphs may be observed. 

\subsection{Patterns}
There are 4 main patterns which may be revealed by reordering the matrix. These patterns may be combined in such a way, that for example a subgraph creates a circle, but one node if it is connected to every other node. This results in a combination of the star and the circle pattern. 
The four different patterns can be seen in the corresponding figures~\ref{fig:patterns}.

\begin{table}[H]
  \centering
\begin{tabu}{cc}
	\includegraphics[width=0.49\textwidth]{images/pattern_star}  &
	\includegraphics[width=0.49\textwidth]{images/pattern_full} \\
	\includegraphics[width=0.49\textwidth]{images/pattern_circle} &
	\includegraphics[width=0.49\textwidth]{images/pattern_line} 
\end{tabu}
\captionof{figure}{4 differnet patterns: star, full, circle and line\label{fig:patterns}}
\end{table}



\cleardoublepage
\chapter{Scalability}
\label{chap:Scalability}


Graph visuaisations are always limited by the number of nodes, aswell as the number of edges. 
The simple geometrie of Adjacenzy matrxes allows an esimation of those upper limits. May edgtes is always nodes**2.

On a typical sreen with ~25” alows to  about 

table

The most simple
 aproach is zooming, unfortunatly this brings new problems, and doeas not solve

Simple zooming bring problems:
Zoomed in:Invisible edges in hidden zoom area, tasks wie mit allen verbunden geht nimma

Simple zooming dows not solve problems:
Zoomed out: to small cols/rows with unreadable labels

Improvments in a spezific kontext, 
Clustering, recursive clustering: relations beteen nodes, clusters (ham). depends on clustering, this is dataproprocessing from my point of view.

Hybrid: dens clusters (nodetrix) is usefull if dense subclusters a present. Shows them as adjm in traditional graph



\section{Matrix subdivision}

\subsection{clusteralgo}
algo -> sort,
multible sorts (within cluser, clusters)

minimizes unseed edges. sugest marks for unseedn edges

\subsection{Transformations}
\subsubsection{Header}
\subsubsection{Cell}





\section{Recursive Matrix subdivision}

additional header stuff





\section{Hybrid Representation}

shows adjm in tratitionsl graph
best for dense clustes in graph



\cleardoublepage
%----------------------------------------------------------------
%
%  File    :  survey-style.tex
%
%  Author  :  Keith Andrews, IICM, TU Graz, Austria
% 
%  Created :  27 May 93
% 
%  Changed :  19 Feb 2004
% 
%----------------------------------------------------------------


\chapter{Cell types}\label{chap:celltypes}


\cleardoublepage
%----------------------------------------------------------------
%
%  File    :  thesis-tech.tex
%
%  Author  :  Eva Rott, TU Graz, Austria
% 
%  Created :  14 May 2017
% 
%  Changed :  16 May 2017
% 
%----------------------------------------------------------------


\chapter{Reordering}\label{chap:reordering}

Reordering describes the process of either moving nodes in a graph or moving rows or columns in a matrix. There are two types of reordering: manual and automatic. Manual reordering is done by the user of a software. This survey focuses on automatic reordering, which is done by a software tool based on its implemented algorithms. The information used as input for the algorithms can be the node label, node in / out degree or clustering data. The mentioned information sources are not a complete list of available reordering input data. However, they were found in the tested programs and they are described in more detail in the following enumeration.

\begin{enumerate}
	\item Reordering using node label. The data used for this example describes a number of functions of a program connected corresponding to the its control flow. Figure $[function_calls_graph.png]$ TODO shows the directed graph with the functions as nodes; Figure $[sample_matrix.png]$ TODO the unsorted matrix representation of the graph. The result of reordering the matrix based on alphabetic label name order can be seen in figure $[reordered_by_label.png]$ TODO.
	
	\item Reordering using node out degree. The data used for this example is the same as in TODO. Figure TODO shows a reordered matrix based on ascending order of node in and out degree. The function with the largest number of incoming links is the rightmost column and the node function with the largest number of outgoing links the lowermost row of the matrix.

	\item Reordering using node clustering. As node clustering needs to be computed first, this type of reordering is explained in an example taken from the program Nodetrix. Displayed in figure TODO $[nodetrix_reordering.png]$ is the graph with the clustered nodes, marked in different colors, and some sub-matrices for some of the ordered clusters.
\end{enumerate}


\cleardoublepage
%----------------------------------------------------------------
%
%  File    :  thesis-tech.tex
%
%  Author  :  Eva Rott, TU Graz, Austria
% 
%  Created :  14 May 2017
% 
%  Changed :  16 May 2017
% 
%----------------------------------------------------------------


\chapter{Matrix headers}\label{chap:headers}

A standard matrix visualization contains a symmetric grid representing the node connections and node labels on top and to the left side of the grid. In this survey this area and in general the area around the grid is referred to as matrix header. The matrix header can be used for visualizing additional information about the matrix data. An example for that is a group of node connections, the node density or the current level of zoom in a multi-layer matrix.

There are various techniques to achieve additional information visualization in a matrix header. Most of the visualization techniques can be found in the program Matrix Explorer. The Matrix Explorer is another tool for matrix visualization of graphs. As seen in figure $[matrixExplorer.png]$ TODO, it displays the matrix without a zoom functionality, while giving a small overview of the full matrix in the top left corner of the graphical user interface. Aside from the matrix visualization, the Matrix Explorer provides various options for filtering or sorting of nodes and executing operations on the matrix headers. The following section describes the most common operations and gives one example per listed tool for a better understanding. 
\begin{enumerate}
	\item Lines in the matrix header. As seen in figure $[path_highlighting.png]$ TODO from the program MatLink, created by Henry et. al $[henry phd 2008 ref!]$ curved lines are used for highlighting node connections. The shortest path is highlighted in red. When one node is selected, the program draws the paths in the headers of the matrix. Using this visual information it can quickly be seen how many other nodes are connected directly to the selected node. In addition, the path from one node to another connected node can be traversed using these lines. TODO $[ref henry 2008 phd]$

	\item Histogram per node in the matrix header. A histogram can be computed over various node properties. An example for such a property is the node degree. The Matrix Explorer offers this histogram functionality. Figure $[better_histo.png]$ TODO shows a matrix visualization of data similar to that used for the first two examples in the previous chapter. Again, a set of functions from a program are represented as nodes and the program’s control flow as links. Looking at the histogram which was computed over the outgoing links it becomes obvious that the main function has the largest light gray area. This means that it has the largest number of outgoing links of all nodes. In contrast to the node out degree distribution, the incoming links follow a more balanced distribution. Considering this example, a histogram in the header is well suited for showing the general distribution of the node degree. In case nodes containing extreme values shoulreorderingd be highlighted, the header color visualization technique can be used.

	\item Colors in the matrix header. Every node in a graph can be assigned a color in a certain range. This color distribution assigned to the nodes can be computed for the same properties as the histogram. The darker the color the higher the value of this node. Considering the same example and figure $[better_color.png]$ TODO as in the previous section, it becomes obvious that the start function has no incoming links as it is the root function of the program.

	\item Histogram per section in the matrix header. In contrast to the histogram computed per node as explained in $[point 2]$ TODO, a histogram can also be computed over multiple columns or rows at the same time. Figure $[matrixzoom_histo.png]$ TODO shows an example. The screenshot was taken from the matrix visualization program MatrixZoom. The displayed matrix is divided into three times three sub-sections. For every group of three vertically or horizontally aligned sections the amount of data contained in it is computed. Next, those values are transformed into a histogram representation, which is then drawn as light blue bars on the right and bottom matrix header. The result shows that the center section of the matrix has the largest density of data points of all sub-sections.

\end{enumerate}

\cleardoublepage
\setcounter{secnumdepth}{-1}
\chapter{Conclusion}
Several tools for matrix visualization of graphs were discussed and evaluated with focus on the implementation of a zoom function, cell visualization techniques and matrix header visualizations. Evaluating all of these three points, it became clear that each of the presented tools implements a different key aspect. While MatrixZoom has the best zoom implementation, Nodetrix provides very good cell visualization techniques, Matrix Explorer offers the best support for additional information visualization. 

All in all, no program can be recommended in general for any application. Best approach is to consider the type and size of the data before choosing a matrix visualization tool.

\cleardoublepage
\printbibliography[heading=bibintoc]

\end{document}

